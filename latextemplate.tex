\documentclass{article}
\usepackage[a4paper, margin=1cm]{geometry}

\usepackage{circuitikz}
\usepackage{siunitx}
\usepackage{wrapfig}

\begin{document}

{\large \textbf{Petit Exercice pour @Prenom@ @Nom@}}

%% \title{Petit exercice}
%% \maketitle
\begin{wrapfigure}{l}{2cm}   
\begin{circuitikz} [american, voltage shift=0.5]
 \draw (0,0)
 to [american voltage source, invert, o-o] (0,3)
 to [short, -*, below, i>_=${i_1=\SI{20}{\mA}}$] (2,3)
 to [R=$R$, i>_=$i_2 10mA$] (2,0) -- (0,0);
 \draw (2,3) -- (4,3)
 to [lamp=$L$, i>_=$i_3 10mA$]
(4,0) to[short, -*] (2,0);
\end{circuitikz}
\end{wrapfigure}

En utilisant les indications du schéma sur la feuille, rapellez la relation entre $i_1$, $i_2$ et $i_3$:

Calculez $i_1$: 

\end{document}
