\documentclass{article}
\usepackage[a4paper, margin=2cm]{geometry}

\usepackage{circuitikz}
\usepackage[french]{babel}
%%\usepackage[autolanguage]{numprint}
\usetikzlibrary{babel} 

\usepackage{siunitx}
\usepackage{wrapfig}

\setlength{\parindent}{0pt}


\begin{document}

{\large \textbf{@CLASS@°@CLNB@ Petit exercice pour @PRENOM@ @NOM@}}
\hfill

\begin{wrapfigure}{l}{0.3\textwidth}
\begin{circuitikz}[european]
 \draw (0,0)
 to [american voltage source, invert, o-o] (0,3)
 to [short, -*, i>^=@IA@] (2,3)
 to [R=$R$, i_>=@IB@] (2,0) -- (0,0);
 \draw (2,3) -- (4,3)
 to [lamp=$L$, i>_=@IC@]
(4,0) to[short, -*] (2,0);
\end{circuitikz}
\end{wrapfigure}

En utilisant les indications du schéma sur la gauche (le dipole doté d'un "+/-" est un générateur de tension), rapellez la relation entre $i_1$, $i_2$ et $i_3$ (attention au sens des flèches!):

Calculez $i_1$: 

\hfill

@SOLSTRING@

\end{document}
