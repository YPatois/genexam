\documentclass[11pt] {article}
\usepackage[paperheight=6.9cm,paperwidth=21cm,hmargin=2cm,vmargin=1cm]{geometry}
\usepackage{titleps}
\usepackage[absolute]{textpos}
\usepackage{xfp}
\usepackage{../../WriteOnGrid}

\usepackage{circuitikz}
\usepackage[french]{babel}
%%\usepackage[autolanguage]{numprint}
\usetikzlibrary{babel} 

\usepackage{siunitx}
\usepackage{wrapfig}

\newpagestyle{ruled} {
  \sethead{}{}{}\headrule
  \setfoot{}{}{}\footrule
}

\pagestyle{ruled}

\setlength{\parindent}{0pt}

\TPGrid[1cm,1cm]{5}{5}

\begin{document}
%\TPShowGrid{5}{5}

%\thispagestyle{empty}

{\large \textbf{Classe: @CLASS@°@CLNB@ -- Petit exercice pour @PRENOM@ @NOM@}}

\begin{wrapfigure}{l}{0.35\textwidth}
\scriptsize
\begin{circuitikz}[european]
 \draw (0,0)
 to [american voltage source, invert, o-o] (0,3)
 to [short, -*, i>^=@IA@] (2,3)
 to [lamp=$L_1$, i>_=@IB@] (2,0) -- (0,0);
 \draw (2,3) -- (4,3)
 @SECOND_ITEM@
\end{circuitikz}

\end{wrapfigure}

En utilisant les indications du schéma sur la gauche (le dipole doté d'un "+/-" est un générateur de tension), rapellez la relation entre $i_1$, $i_2$ et $i_3$ (attention au sens des flèches!):
\begin{flushright}
\begin{EnvQuadrillage}[NbCarreaux=14x1,Grille=Seyes,Marge=1]
\end{EnvQuadrillage}
\end{flushright}

Calculez $i_1$: 
\begin{flushright}
\begin{EnvQuadrillage}[NbCarreaux=14x1,Grille=Seyes,Marge=1]
\end{EnvQuadrillage}
\end{flushright}


\begin{textblock}{1}(2,5)
    \texttt{\tiny Serial Number: @SOLSTRING@}
\end{textblock}

\pagebreak
\thispagestyle{empty}

{\large \textbf{Classe de @CLASS@°@CLNB@}}
\hfill

{\Large \textbf{Petit exercice pour}}
\hfill

\begin{center}
{\Huge \textbf{@PRENOM@ @NOM@}}
\end{center}

\end{document}
